\documentclass[11pt]{article}
\usepackage{ifthen}
\usepackage{graphicx}
\usepackage{cite}
\usepackage{hevea}
\input{synrc.hva}
%HEVEA \loadcssfile{synrc.css}
\begin{document}
%HEVEA \rawhtmlinput{templates/head-hevea.htx}

\title{Synrc Labs}
\author{Maxim Sokhatsky}

%HEVEA \begin{divstyle}{selectedwrapper}
%HEVEA \begin{divstyle}{wrapper}
%HEVEA \begin{divstyle}{threecol}

\section*{Directory Server}
%HEVEA \begin{divstyle}{block}
\paragraph{}
Directory Server is a distributed storage that can handle tens of
millions online user requests. It is basically an object data storage
with proven LDAP as front-end interface protocol.
In its core as most of LDAP servers Synrc Directory is based on
key-value distributed data store, DHT.
\paragraph{}
It can be integrated as in pure Erlang environments as within an
existed heterogeneous network with message queues, broker servers
and services that rely on access to LDAP directories, such as XMPP,
auth services, directory books, mobile sync, etc.

Sources: \footahref{https://github.com/synrc/directory-server}{synrc.com/server/directory}
%HEVEA \end{divstyle}


\section*{Sync Server}
%HEVEA \begin{divstyle}{block}
\paragraph{}
Sync Server is written in Erlang SyncML server that can handle
millions of active users. It is very small and managable.
Best using in a couple with Directory.

Sources: \footahref{https://github.com/synrc/sync-server}{synrc.com/server/sync}
%HEVEA \end{divstyle}

\section*{Chat Server}
%HEVEA \begin{divstyle}{block}
\paragraph{}
Chat Server is a modification of ejabberd XMPP server for use with MongoDB backend.

Sources: \footahref{https://github.com/synrc/chat-server}{synrc.com/server/chat}
%HEVEA \end{divstyle}


%HEVEA \end{divstyle}
%HEVEA \begin{divstyle}{threecol}



\section*{\footahref{http://synrc.com/client/chat/haiku}{Haiku Chat}}
%HEVEA \begin{divstyle}{block}
\paragraph{}
Haiku Chat is tiny, about 300KB XMPP client.
It supports core XMPP protocol, multi-user chat, Google accounts,
Psi bookmarks, In-band registration and other features.
It is written for Haiku, free open-source operating system inspired by BeOS.

Sources: \footahref{http://github.com/synrc/haiku-chat-client}{synrc.com/client/chat/haiku}
%HEVEA \end{divstyle}


\section*{\footahref{http://synrc.com/client/sync/windows}{Windows Sync}}
%HEVEA \begin{divstyle}{block}
\paragraph{}
Synrc Contacts Application is about organize,
import/export, sync your Address Books with Google Contacts,
Microsoft Outlook, NOKIA Phones, Windows Contacts local folder,
LDAP Directory, Yahoo! Contacts and Windows Live Contacts.
\paragraph{}
Synrc Contacts is "buddhist view" zen Application for
Vista and Windows 7. It's simplified design with one button
makes you life easy with managing personal contact information.

Sources: \footahref{https://github.com/synrc/windows-sync-client}{synrc.com/client/sync/windows}
%HEVEA \end{divstyle}

\section*{Symbian Chat}
%HEVEA \begin{divstyle}{block}
\paragraph{}
Symbian Chat is tiny Qt client. It supports core XMPP protocol,
multi-user chat, Google accounts, Psi bookmarks, In-band registration
and other features. It is written for Symbian formerly known as EPOC using Qt framework.

Sources: \footahref{https://github.com/synrc/symbian-chat-client}{synrc.com/client/chat/symbian}
%HEVEA \end{divstyle}


%HEVEA \end{divstyle}
%HEVEA \begin{divstyle}{threecol last}

\section*{Category Theory}
%HEVEA \begin{divstyle}{block}
\paragraph{}
Here is a set of Books on Category Theory, Dependent Types, Lambda and Process Calculus.

Sources: \footahref{http://synrc.com/publications/cat}{synrc.com/publications/cat}
%HEVEA \end{divstyle}

\section*{\footahref{http://synrc.com/research/io/index.htm}{Io}}

%HEVEA \begin{divstyle}{block}
\paragraph{}
As some of us are fans of Smalltalk and LISP, we cannot pass the
beautiful syntaxic forth start-up. Synrc Io for CLR --- one of simplest but very
powerful in potential languages. Inpired by Smalltalk it's simplified
syntax free from syntax trash. We support Open Source implementation
of Io Language for Common Language Runtimes such as Mono and .NET.

Sources: \footahref{http://github.com/synrc/research-io}{synrc.com/research/io}
%HEVEA \end{divstyle}

%\section*{PL/1}
% EVEA \begin{divstyle}{block}
%\paragraph{}
%Synrc PL/1 for CLR is our first variant of PL/1 version for .NET.

%Sources: \footahref{http://github.com/synrc/research-pl}{synrc.com/research/pl}
% EVEA \end{divstyle}

\section*{ML}
%HEVEA \begin{divstyle}{block}
\paragraph{}
Synrc ML written in C and is tiny ML byte-code compiler that incororates CAM machine
along with process calculus and lazy evaluations.

Sources: \footahref{http://github.com/synrc/research-ml}{synrc.com/research/ml}
%HEVEA \end{divstyle}

\section*{\footahref{beos/beos.htm}{BeOS}}
%HEVEA \begin{divstyle}{block}
\paragraph{}
Synrc Research Center archived all information about BeOS operating system 
including CD images and have dedicated page for this purposes.
%HEVEA \end{divstyle}


%HEVEA \end{divstyle}
%HEVEA \end{divstyle}
%HEVEA \end{divstyle}
%HEVEA \begin{divstyle}{clear}{~}\end{divstyle}

%HEVEA \footerfalse
%HEVEA \rawhtmlinput{templates/foot.htx}

\end{document}
