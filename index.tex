\documentclass[11pt]{article}
\usepackage{ifthen}
\usepackage{graphicx}
\usepackage{cite}
\usepackage{hevea}
\input{synrc.hva}
%HEVEA \loadcssfile{synrc.css}
\begin{document}

\title{Home}
\author{Maxim Sokhatsky}

%HEVEA \begin{divstyle}{selectedwrapper}
%HEVEA \begin{divstyle}{wrapper}
%HEVEA \begin{divstyle}{threecol}

\section*{Erlang}
%HEVEA \begin{divstyle}{block}
\paragraph{}
Erlang was created in 1987 in the Ericsson Computer Science Laboratory.
It is the implementation language of such high-load services as Amazon Web Services,
Facebook Chat and Ericsson switches.
%HEVEA \end{divstyle}

%HEVEA \end{divstyle}
%HEVEA \begin{divstyle}{threecol}

\section*{Products}
%HEVEA \begin{divstyle}{block}
\paragraph{}
Our products support millions of active connections thanks to Erlang wich is
an uncompromising platform for high-loaded applications.
Erlang versions of the LDAP and SyncML servers is our main product line.
%HEVEA \end{divstyle}
%HEVEA \end{divstyle}
%HEVEA \begin{divstyle}{threecol last}

\section*{Services}
%HEVEA \begin{divstyle}{block}
\paragraph{}
As an Erlang consulting company we provide a set of services starting from friendly advice to
consulting and full development cycle according to your business.
Our main development rule is ``No Bullshit''.
%HEVEA \end{divstyle}

%HEVEA \end{divstyle}
%HEVEA \end{divstyle}
%HEVEA \end{divstyle}
%HEVEA \begin{divstyle}{clear}{~}\end{divstyle}

%HEVEA \begin{divstyle}{nonselectedwrapper}
%HEVEA \begin{divstyle}{verywidecol}
\section*{Frameworks}
\paragraph{}
Our success would be impossible without the support of widely accepted tools for the Erlang/OTP platform.
Due to the existence of RabbbitMQ, GProc, Riak, ejabberd, exmpp, Nitrogen and
Cowboy we were able to deliver state of the art solutions to our customers. 
Our secret is beneath the way of what we are using and how.

\footahref{research.htm}{Lay cards on the Table}

%HEVEA \end{divstyle}
%HEVEA \end{divstyle}
%HEVEA \begin{divstyle}{clear}{~}\end{divstyle}



\end{document}
