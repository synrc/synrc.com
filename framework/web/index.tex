\documentclass[11pt]{article}
\usepackage{ifthen}
\usepackage{graphicx}
\usepackage{cite}
\usepackage{hevea}
\input{synrc.hva}
%HEVEA \loadcssfile{../../synrc.css}

\begin{document}

\title{N2O}
\author{Maxim Sokhatsky}

%HEVEA \begin{divstyle}{nonselectedwrapper}
%HEVEA \begin{divstyle}{article}
%HEVEA \begin{divstyle}{smallcol}
\includeimage{../../images/n2o.png}
%HEVEA \end{divstyle}
%HEVEA \begin{divstyle}{articlecol}
\section*{N2O: Fastest Web Framework for Erlang}
\subsection*{Binary events over WebSockets}

Nitrogen 2 Optimized. Binary page construction.
Binary data transfer. Websocket async interface for
page updates. No processes spawn, works within
Cowboy processes. Page render is several times faster than
original Nitrogen.

\subsection*{Why Erlang in Web ?}
We've measured all existing modern web frameworks with latests functional languges and Cowboy still the king.
You can see raw HTTP performance of functional and C-like languages with concurrent primitives:

%HEVEA \rawhtmlinput{templates/webservers.htx}

We outperform full Nitrogen stack with only 2X downgrade of raw HTTP Cowboy
performance thus rise rendering performance several times in compare to
any other functional web framework and for sure it is faster than raw HTTP node.js performance.

%HEVEA \end{divstyle}

%HEVEA \begin{divstyle}{toc last}
\section*{TOC}
\paragraph{}
\footahref{}{Overview} \@br
\footahref{}{1. Architecture} \@br
\footahref{}{2. Rendering} \@br
\footahref{}{3. Push} \@br
\footahref{}{4. Samples} \@br
%HEVEA \end{divstyle}

%HEVEA \end{divstyle}
%HEVEA \end{divstyle}
%HEVEA \begin{divstyle}{clear}{~}\end{divstyle}

\end{document}

