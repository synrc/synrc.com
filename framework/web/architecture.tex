\documentclass[11pt]{article}
\usepackage{ifthen}
\usepackage{graphicx}
\usepackage{cite}
\usepackage{hevea}
\input{../../synrc.hva}
%HEVEA \loadcssfile{../../synrc.css}

\begin{document}

\title{N2O: Architecture}
\author{Maxim Sokhatsky}

%HEVEA \begin{divstyle}{nonselectedwrapper}
%HEVEA \begin{divstyle}{article}
%HEVEA \begin{divstyle}{smallcol}
\paragraph{}
\includeimage{../../images/n2o.png}
%HEVEA \end{divstyle}
%HEVEA \begin{divstyle}{articlecol}
\section*{N2O: Architecture}

\subsection*{Reduced Latency}
The secret of reduced latency is simple. We try to deliver rendered HTML
as soon as possible and render JavaScript only after WebSocket initialization.
We use three steps and three erlang processes for achieve that.

%HEVEA \rawhtmlinput{templates/page-lifetime.htx}

\subsection*{HTTP process}
In the first HTTP handler we only render HTML and all created javascript actions are stored in transition process.

%HEVEA \rawhtmlinput{templates/transition.htx}

HTTP handler dies immediately after returning HTML. Transition process
waits for request from WebSocket handler.

\subsection*{Transition process}
Just after receiving HTML browser initiates WebSocket connection
and this starts WebSocket handler on the server. After returning
javascript actions transition process dies and only WebSocket handler stays.
At this point initialization phase is done.

\subsection*{WebSocket process}
After that all client / server communication is done via WebSocket channel.
All events that come from the browser are handeled by N2O, which renders elements
to HTML and actions to JavaScript.

%HEVEA \rawhtmlinput{templates/disqus.htx}

%HEVEA \end{divstyle}

%HEVEA \begin{divstyle}{toc last}

%HEVEA \begin{divstyle}{toc}
\section*{TOC}
\paragraph{}
\footahref{http://synrc.com/framework/web/}{Overview} \@br
\footahref{http://synrc.com/framework/web/setup.htm}{1. Setup} \@br
\footahref{http://synrc.com/framework/web/architecture.htm}{2. Architecture} \@br
\footahref{http://synrc.com/framework/web/api.htm}{5. API} \@br
\footahref{http://synrc.com/framework/web/elements.htm}{3. Elements} \@br
\footahref{http://synrc.com/framework/web/actions.htm}{4. Actions} \@br
\footahref{http://synrc.com/framework/web/macros.htm}{6. JavaScript} \@br

\footahref{http://synrc.com/framework/web/n2o/doc/book.pdf}{Download PDF} \@br
%HEVEA \end{divstyle}


%HEVEA \end{divstyle}

%HEVEA \end{divstyle}
%HEVEA \end{divstyle}
%HEVEA \begin{divstyle}{clear}{~}\end{divstyle}

\end{document}

